\documentclass[11pt]{article}

\usepackage{amsmath}
\usepackage{amssymb}
\usepackage{enumerate}
\usepackage{enumitem}
\usepackage{booktabs}
\usepackage{csquotes}
\usepackage[margin=2cm]{geometry}
\usepackage{hyperref}
\usepackage{tabularx}
\usepackage{tikz}
\usetikzlibrary{patterns, shapes.geometric, positioning, bayesnet}

\usepackage{titling}
\setlength{\droptitle}{-7em}
\usepackage{titlesec}
\titlespacing\section{0pt}{12pt plus 4pt minus 2pt}{4pt plus 2pt minus 2pt}

\title{Project Proposal \vspace{-1em}}

\author{CS396 Causal Inference}

\begin{document}

\maketitle

\section{Group members}

Ella Jones, Alexander Romanenko, Ella Cutler, Catherine Tawadros

\section{Problem Statement}

Our project aims to examine the potential causal pathways through which yield curve inversions impact GDP growth, particularly focusing on the role of economic fear as a mediating factor. Yield curve inversions are widely recognized as a signal of potential economic downturns; however, their direct impact on GDP growth might be less straightforward and more mediated through investor, business, and consumer confidence levels.

We hypothesize that yield curve inversions, by themselves highly publicized indicators, may not directly cause changes in GDP but instead influence GDP through induced fear and uncertainty in economic agents. This project will explore how 1) investor sentiment, 2) business confidence, and 3) consumer confidence, could serve as intermediaries through which yield curve inversions lead to GDP declines in subsequent periods. Metrics such as the Volatility Index (VIX), Business Confidence Index (BCI), and Consumer Confidence Index (CCI), will serve as proxies for the level of economic fear and uncertainty.

The interaction between these fear indicators and yield curve inversions is crucial for understanding the effects on GDP growth. By integrating these variables into our analysis, we aim to uncover whether the fear generated by yield curve inversions has a significant causal effect on economic performance in subsequent periods. This understanding is vital for policymakers and economic analysts who use such indicators to forecast economic health and make informed fiscal and monetary decisions.

\section{Causal Questions or Hypotheses}

Our project seeks to investigate the causal relationship between yield curve inversions, economic fear indicators, and GDP growth. We formulate our causal hypothesis as follows:

\begin{enumerate}[label={(\alph*)}]
\item \textbf{Treatment(s) and Outcome(s):} 
  The primary treatment variable in our study is the occurrence of a yield curve inversion, operationalized as binary (1 if the yield curve is inverted at time \( t \), 0 otherwise). The outcome variables include GDP growth over the subsequent quarters (6 months and 12 months post-inversion). We also consider economic fear indicators (VIX, BCI, CCI) as secondary treatments or mediators that might influence the primary relationship between yield curve inversion and GDP growth.

\item \textbf{Framing the Question as a Contrast of Counterfactual Random Variables:} 
  The core of our causal question can be framed as follows: What would the GDP growth rate be 6 and 12 months after a yield curve inversion (\( Y_{t+6} \), \( Y_{t+12} \)) compared to what it would have been had the inversion not occurred (\( Y_{t+6}(0) \), \( Y_{t+12}(0) \))? This counterfactual setup mimics a hypothetical randomized trial where the market conditions are randomly assigned to experience a yield curve inversion or not, allowing us to isolate the effect of the inversion from other concurrent economic conditions.

\item \textbf{Potential Treatments and Outcomes Discussion:}
  While our primary interest lies in yield curve inversions, considering the economic fear indicators (VIX, BCI, CCI) as secondary treatments or mediators provides a more nuanced understanding of the causal pathway. Arguments for this approach include the potential for these indicators to capture the psychological and behavioral responses of investors and consumers to economic signals, which could in turn affect GDP growth. Alternatively, focusing solely on the direct impact of yield curve inversions on GDP, without accounting for these mediating effects, could oversimplify the complex dynamics at play in economic systems and potentially overlook significant intermediary processes.
\end{enumerate}

\section{Dataset(s)}

\begin{enumerate}[itemsep=0em,label={(\alph*)}]
\item \textbf{Dataset Background and Contents:} The Federal Reserve Economic Data (FRED) provides historical data on various aspectcs of economic activity, primarily in the United States. Using FRED's API, we collected historical data from 1990 to the present on the following categories: Real Gross Domestic Product, Real Government Consumption Expenditures and Gross Investment, Real Gross Private Domestic Investment, Real Net Exports of Goods and Services, 10-Year Treasury Constant Maturity Minus 2-Year Treasury Constant Maturity (the Yield Curve), the CBOE Volatility Index, the Business Confidence Index, the Consumer Confidence Index, the Unemployment Rate, the Federal Funds Effective Rate, and Sticky Price Consumer Price Index less Food and Energy.
\item \textbf{Limitations:} The data has several limitations:
  \begin{enumerate}[itemsep=0em,label={\roman*.}]
      \item Frequencies of observations might not be consistent across each data category. Our dataset contains four categories with quarterly observations, four categories with monthly observations, and two categories with daily observations. We are considering standardizing all data to a monthly frequency using interpolation and aggregation techniques, but these specific techniques are still unknown and each presents its own set of limitations.
      \item Missing values can be found in some of our data which might be problematic for modeling techniques that do not accomodate for this. We may be required to use interpolation techniques to fill-in missing values. 
      \item The data formatting options offered by FRED vary significantly, potentially requiring us to perform our own scaling and transformations to ensure consistency. Furthermore, data presented as percent changes might benefit from logarithmic transformations.
      \item Some of our time-series data may be nonstationary. We intend to use a binary indicator for yield curve inversions (taking values of 1 or 0), which is inherently nonstationary. Additionally, VIX values are likely to be nonstationary due to non-constant mean and variance. Nonstationary time-series could limit the effectiveness of certain modeling approaches.
      \item Subsequent analysis could be limited by confounding variables or mediators, which are not included in our initial dataset.
  \end{enumerate}
\item \textbf{Format:} Each data category is represented as its own time-series with daily, monthly, or quarterly frequencies. Data is extracted from 1990 to 2023, which would provide us with 396 observations if all categories of data were formatted into monthly observations using interpolation and aggregation techniques. Therefore, our dataset could be formatted as a 396x12 ($N\times D$) matrix, with all features being continuous variables, with the exception of our yield curve inversion indicator which would be a binary variable. 
\item \textbf{DataFrame Preview:} The first 6 rows of our pandas DataFrame are included below (average values were used to convert daily frequency values of VIX to monthly values):
\begin{verbatim}
DATE        YC_INV RGDP RGCI RGPDI RNE  VIX BCI CCI UNRATE FEDFUNDS M/M_CPI   
1990-01-31    0   10047 2535 1304  -64  23 99.8 100.7  5.4    8.23    0.45
1990-02-28    0    NA    NA   NA   NA   23 99.0 100.6  5.3    8.24    0.38
1990-03-31    0    NA    NA   NA   NA   20 99.2 100.7  5.2    8.28    0.63
1990-04-30    0   10084 2539 1305  -54  21 99.2 100.8  5.4    8.26    0.43
1990-05-31    0    NA    NA   NA   NA   18 99.2 100.6  5.4    8.18    0.37
1990-06-30    0    NA    NA   NA   NA   17 99.9 100.3  5.2    8.29    0.59
...
\end{verbatim}

\item \textbf{Variable Overview:} FRED provides access to multiple categories of historical data that we are interested in. All data is continuous with the exception of yield curve inversions which are represented with a binary indicator:
  \begin{enumerate}[itemsep=0em,label={\roman*.}]
    \item \href{https://fred.stlouisfed.org/series/GDPC1}{\underline{\textcolor{blue}{Real GDP}}}: Measures the inflation adjusted value of the goods and services produced by labor and property within the United States. The dataset provides quarterly values (seasonally adjusted) beginning in 1947.
    \item \href{https://fred.stlouisfed.org/series/GCEC1}{\underline{\textcolor{blue}{Real Government Consumption Expenditures and Gross Investment}}}: Measures the inflation adjusted value of GDP that is accounted for by the government sector. The dataset provides quarterly values (seasonally adjusted) beginning in 1947.
    \item \href{https://fred.stlouisfed.org/series/GPDIC1}{\underline{\textcolor{blue}{Real Gross Private Domestic Investment}}}: Measures inflation adjusted private investment in the United States. It includes business expenditures for things like machines, tools, land, buildings, and changes in inventories. The dataset provides quarterly values (seasonally adjusted) beginning in 1947.
    \item \href{https://fred.stlouisfed.org/series/NETEXC}{\underline{\textcolor{blue}{Real Net Exports of Goods and Services}}}: Measures the inflation adjusted value of total exports minus total imports in the United States. The dataset provides quarterly values (seasonally adjusted) beginning in 1970.
    \item \href{https://fred.stlouisfed.org/series/T10Y2Y}{\underline{\textcolor{blue}{10-Year Treasury Constant Maturity Minus 2-Year Treasury Constant Maturity}}}: Records spreads between 10-Year Treasury Constant Maturity and 2-Year Treasury Constant Maturity. When the spread is below 0, this is known as a yield curve inversion. The dataset provides daily values beginning in 1976.
    \item \href{https://fred.stlouisfed.org/series/VIXCLS}{\underline{\textcolor{blue}{CBOE Volatility Index (VIX)}}}: Measures market expectation of near term volatility conveyed by stock index option prices, with values used as a way to gauge market sentiment. Investors are descrbed as being more fearful when VIX levels are higher than normal. The dataset provides daily values beginning in 1990.
    \item \href{https://fred.stlouisfed.org/series/BSCICP03USM665S}{\underline{\textcolor{blue}{Business Confidence Index}}}: Measures business confidence levels in the United States through surveys. Businesses are descrbed as being more fearful when sentiment levels are lower than normal. The dataset provides monthly values (seasonally adjusted) beginning in 1950.
    \item \href{https://fred.stlouisfed.org/series/CSCICP03USM665S}{\underline{\textcolor{blue}{Consumer Confidence Index}}}: Measures consumer confidence levels in the United States through surveys. Consumers are descrbed as being more fearful when sentiment levels are lower than normal. The dataset provides monthly values (seasonally adjusted) beginning in 1960.
    \item \href{https://fred.stlouisfed.org/series/UNRATE}{\underline{\textcolor{blue}{Unemployment Rate}}}: Records the number of people unemployed as a percentage of the labor force. The dataset provides monthly values (seasonally adjusted) beginning in 1948.
    \item \href{https://fred.stlouisfed.org/series/FEDFUNDS}{\underline{\textcolor{blue}{Federal Funds Effective Rate}}}: Records interest rates banks charge each other to borrow or lend excess reserves. The Federal Funds Rate is typically raised to temper economic activity if the Federal Reserve believes the economy is growing too fast and there are existing inflationary pressures. In periods of lower economic activity, the Federal Reserve may set a lower federal funds rate target to encourage increased economic growth. This dataset includes monthly values beginning in 1954.
    \item \href{https://fred.stlouisfed.org/series/CORESTICKM157SFRBATL}{\underline{\textcolor{blue}{Sticky Price Consumer Price Index less Food and Energy}}}: Measures inflation on a subset of goods and services where price changes infrequently. It is preferred over the general Consumer Price Index in informing Monetary Policy. This dataset includes month over month percent changes in prices (seasonally adjusted) beginning in 1967.
  \end{enumerate}
\end{enumerate}

ii. What are the possible1 causal relationships between this variable and the other variables (in part e)?
(f) What is at least one variable that your dataset doesn’t contain but might be a causal factor? How might such a variable complicate your causal question(s)?



\section{Expectations and Concerns}

Through this project, we aim to deepen our understanding of the causal relationships between economic indicators and GDP growth, with a particular focus on the impact of yield curve inversions mediated by economic fear indicators such as the VIX, BCI, and CCI sentiment indices. This investigation is not only academically stimulating but also critically relevant to economic policy-making and forecasting. By utilizing causal inference methodologies learned in CS396, we hope to elucidate the complex interplay among these economic variables and their collective influence on economic cycles.

One of our primary objectives is to employ advanced statistical and econometric methods, including difference-in-differences (DiD), instrumental variables (IV), propensity score matching, Vector Autoregression (VAR), Vector Error Correction Models (VCEM), dynamic regression models, and Granger Causality tests. These approaches will enable us to handle potential confounders and biases systematically, aiming for more precise and dependable causal estimates.

However, several challenges loom on the horizon. High-dimensional data and missing values pose significant analytical hurdles, potentially compromising the integrity and accuracy of our results. The intricate nature of economic relationships might also complicate the task of isolating the impact of specific variables, particularly the indirect effects of yield curve inversions through fear and uncertainty. Additionally, the validity of our causal interpretations must be continuously scrutinized due to the observational nature of our data. To address these challenges, we plan to implement rigorous data preprocessing, validate our models extensively, and engage continuously with economic literature to refine our assumptions and interpretations. Ultimately, this project is expected to not only enhance our academic prowess but also contribute valuable insights into the predictive capabilities of economic indicators in signaling and influencing real-world economic outcomes.


\section{References}

\begin{itemize}
  \item Arellano, Cristina, and Yan Bai. ``Rare Disasters, Financial Development, and Sovereign Debt.'' \emph{Journal of Political Economy}, vol. 128, no. 7, 2020, pp. 2900-2944. \url{https://www.journals.uchicago.edu/doi/abs/10.1086/707766}
  \item Krishnamurthy, Arvind, and Tyler Muir. ``How Credit Cycles Across a Financial Crisis.'' \emph{NBER Working Paper}, No. 23850, 2019. \url{https://papers.ssrn.com/sol3/papers.cfm?abstract_id=3333517}

  \item Federal Reserve Economic Data (FRED). \emph{Federal Reserve Bank of St. Louis}. Accessed on [today's date]. \url{https://fred.stlouisfed.org/}
\end{itemize}


\end{document}
